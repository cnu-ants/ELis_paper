\documentclass[sigplan,10pt]{acmart}

\settopmatter{printfolios=true,printccs=false,printacmref=false}

% minted 패키지 불러오기
\usepackage{caption}
\usepackage{minted}
\captionsetup{compatibility=false} % minted 캡션 옵션 충돌 방지
\usepackage[utf8]{inputenc}
\usepackage{tabularx}   % 표 열 너비 조절용
\usepackage{xcolor}  % 색상 지정용 (선택)
\usepackage[skip=5pt]{caption}  % skip 값으로 caption과 표 간의 간격 조정
\setlength{\belowcaptionskip}{0pt} % 표와 본문 간격 줄이기
% 표 or listing 과 본문 사이 간격 조절
\captionsetup[table]{skip=3pt}       % 표 내부 캡션 간격
\setlength{\abovecaptionskip}{3pt}   % 캡션 위 간격
\setlength{\belowcaptionskip}{3pt}   % 캡션 아래 간격
\setlength{\textfloatsep}{3pt}       % 본문과 float(listing/table) 사이 간격
\setlength{\floatsep}{4pt}           % float와 float 사이 간격
\setlength{\intextsep}{3pt}          % 본문 중간에 들어가는 float와 본문 사이 간격

\PassOptionsToPackage{colorlinks=true, linkcolor=blue}{hyperref}
\usepackage{hyperref}
\newcommand{\listingname}{Listing} % (필요 시 한글로도 가능)
\def\lstlistingautorefname{\listingname}

\begin{document}

\title{Removing virtual stack in Emulation style LLVM-IR}

\author{Yujin An}
\affiliation{
  \institution{Chungnam National University}
  \country{Daejeon, South Korea}
}
\email{yujina@o.cnu.ac.kr}

\author{Sungho Lee}
\affiliation{
  \institution{Chungnam National University}
  \country{Daejeon, South Korea}
}
\email{eshaj@cnu.ac.kr}

\begin{abstract}
A binary lifter translates binary code into an intermediate representation
(IR), such as LLVM IR. Recent studies have demonstrated that modern binary
lifters can generate semantically correct IR even for complex binaries. In this
study, we focus on two types of LLVM IR generated by binary lifters: high-level
IR (HIR) and emulation-style IR (EIR). While HIR offers advantages for
analysis, EIR achieves higher functional accuracy. However, due to its
structural characteristics, EIR is less suitable for analysis. To address this
trade-off, we propose two methods for transforming EIR into HIR, aiming to
combine the functional accuracy of EIR with the analysis suitability of HIR.
\end{abstract}

\maketitle

\section{Introduction}
Static analysis of binaries is inherently challenging because the compilation
process discards most high-level program information, such as variable and
function names, type definitions, and source-level control structures. Binary
lifters translate binaries into an intermediate representation (IR) to address
this limitation, and they have consequently attracted significant attention.

Among various forms of IR, many lifteres target the LLVM Intermediate
Representation(LLVM IR), a low-level intermediate code within the LLVM compiler
infrastructure due to its extensive optimization and transformation
capabilities. However, binary lifters do not always generate IR that is equally
suitable for program analysis. In general, binary lifters produce two distinct
styles of IR: high-level IR (HIR) and emulation-style IR (EIR). 

A recent study evaluated four LLVM IR based binary lifters. In that study, each
binary was lifted by all four tools and evaluated in terms of analysis
suitability and functional correctness. The results varied significantly
depending on the IR style. HIR demonstrated strong analysis suitability but
lower functional correctness, whereas EIR achieved higher functional
correctness, but was less suitable for analysis.

\begin{listing}[ht]
\begin{minted}[linenos, numbers=left, frame=lines, framesep=2mm, xleftmargin=10pt, fontsize=\footnotesize]{llvm}
define i32 @main() {
  %a = alloca i32, align 4
  %b = alloca i32, align 4
  store i32 99, i32* %a, align 4
  store i32 9000, i32* %b, align 4
  %0 = load i32, i32* %a, align 4
  %1 = load i32, i32* %b, align 4
  %sum = add nsw i32 %0, %1
}
\end{minted}
\caption{High-level style LLVM-IR (Clang generated)}
\label{lst:hir}
\end{listing}

\begin{listing}[ht]
\begin{minted}[linenos, numbers=left, frame=lines, framesep=2mm, xleftmargin=10pt, fontsize=\footnotesize]{llvm}
define i32 @Func_main(i32 %arg_esp) {
  %tmp1 = add i32 %arg_esp, -8
  %1 = inttoptr i32 %tmp1 to i32*
  store i32 99, i32* %1, align 4
  %tmp2 = add i32 %arg_esp, -12
  %2 = inttoptr i32 %tmp2 to i32*
  store i32 9000, i32* %2, align 4
  %3 = load i32, i32* %1, align 4 
  %tmp3 = add i32 %3, 9000
}
\end{minted}
\caption{Emulation style LLVM-IR (Binrec generated)}
\label{lst:eir}
\end{listing}

As shown in \autoref{lst:hir} and \autoref{lst:eir}, EIR emulates machine code more closely through the use of a
virtual stack, while HIR preserves more semantic information such as type
annotations and explicit local variables. Because the virtual stack mimics
program behavior, EIR preserves functional correctness effectively. However,
the absence of explicit variables and other high-level information reduces its
suitability for analysis.

In this study, we propose a method for transforming EIR into HIR. Although EIR
preserves program semantics well due to its emulated stack, this same structure
obscures the high-level information required for analysis. Our goal is to
eliminate the virtual stack in EIR and transform the code into HIR while
preserving semantics, thereby combining the functional correctness of EIR with
the analysis suitability of HIR.

For this purpose, we selected BinRec, a dynamic lifter, to generate EIR for
transformation. We suggest two methods: 1) Variable recover, 2) Recovery of
types

\section{Approach 1: Variable Recovery}
In EIR generated by BinRec, explicit variables do not exist because the lifter
merely emulates the program’s behavior using a stack. As BinRec is a dynamic
lifter, the executed program paths are directly reflected in the lifted code.
BinRec employs a virtual stack and a variable functioning as a virtual stack
pointer (similar to `esp`), emulating the program using both.
As shown in Fig. , a recurring pattern emerges in the emulation of variables:
the lifter first retrieves the virtual stack address and then stores a value to
that address. For example, \%tmp0 and \%1 exhibit a relationship with the
virtual stack pointer. To detect such relationships between variables, we
designate the virtual stack pointer as symbolic and perform symbolic execution.
This allows us to determine which variables are associated with which offsets
relative to the virtual stack pointer. If multiple variables share the same
offset with respect to the virtual stack pointer, we infer that they reference
the same memory location. In such cases, we consolidate them into a single
variable, thereby simplifying the representation (see example results).
However, this approach has limitations inherent to symbolic execution. Since
symbolic execution is a static analysis technique, it suffers from the path
explosion problem: when the number of possible execution paths becomes too
large, it is infeasible to resolve all relationships between variables. This
issue arises when there are multiple control-flow paths. To mitigate this, we
restrict symbolic execution to the basic block level rather than the
function level, ensuring that only a single path is explored per execution. At
present, this method is effective only for relatively simple cases, but we plan
to extend it to handle more complex scenarios.

\section{Approach 2: Type recovery}
Type recovery requires prior variable recovery. In BinRec, however, all type
information is lost during lifting. This is because BinRec, when emulating the
program’s behavior, assigns the same type (i32) to all variables, as the
virtual stack itself is represented as i32.
As noted earlier, in the lifted code there are explicit store operations. By
analyzing these stores, we can infer the type of each variable based on the
values assigned to it. For instance, if a variable of type i32 stores an i32
value, it is treated as an integer variable. Conversely, if it stores the
address of another variable, it is inferred to be a pointer.
To systematically capture these relationships, we construct a points-to graph
in which each edge encodes the target of a variable’s reference. For example,
an i32 variable points to an i32 value, an i32* variable points to an i32
variable, and if the reference is to a value rather than another variable, the
node is treated as a leaf in the points-to graph. By analyzing the
relationships in this graph, we recover the types of variables.
This approach has limitations. Since BinRec supports only primitive types
(e.g., i32, i8), it does not explicitly represent array types. In contrast,
high-level IRs explicitly encode array types, making them easier to identify.
In BinRec, distinguishing array types is non-trivial. As part of our future
work, we plan to extend our variable relationship analysis to enable recovery
of array types.

\section{Conclusion}
In this work, we investigated the transformation of emulation-style LLVM IR
(EIR) into high-level LLVM IR (HIR) to combine the functional correctness of
EIR with the analysis suitability of HIR. Our study focused on eliminating the
virtual stack in BinRec-generated EIR while preserving the original program
semantics. To achieve this, we proposed two approaches: (1) variable recovery
using symbolic execution, and (2) type recovery using points-to graph analysis.
In future work, we plan to extend our framework to recover array types,
function structures, and other high-level constructs, thereby enabling complete
removal of the virtual stack. The ultimate goal is to achieve analysis
suitability comparable to that of high-level style IR, without sacrificing
functional correctness.


\bibliographystyle{ACM-Reference-Format}
\bibliography{acmart}

\end{document}

