\section{Approach 2: Type recovery}
Type recovery requires prior variable recovery. In BinRec, however, all type
information is lost during lifting. This is because BinRec, when emulating the
program’s behavior, assigns the same type (i32) to all variables, as the
virtual stack itself is represented as i32.
As noted earlier, in the lifted code there are explicit store operations. By
analyzing these stores, we can infer the type of each variable based on the
values assigned to it. For instance, if a variable of type i32 stores an i32
value, it is treated as an integer variable. Conversely, if it stores the
address of another variable, it is inferred to be a pointer.
To systematically capture these relationships, we construct a points-to graph
in which each edge encodes the target of a variable’s reference. For example,
an i32 variable points to an i32 value, an i32* variable points to an i32
variable, and if the reference is to a value rather than another variable, the
node is treated as a leaf in the points-to graph. By analyzing the
relationships in this graph, we recover the types of variables.
This approach has limitations. Since BinRec supports only primitive types
(e.g., i32, i8), it does not explicitly represent array types. In contrast,
high-level IRs explicitly encode array types, making them easier to identify.
In BinRec, distinguishing array types is non-trivial. As part of our future
work, we plan to extend our variable relationship analysis to enable recovery
of array types.
