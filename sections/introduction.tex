\section{Introduction}
LLVM Intermediate Representation (LLVM IR) is a low-level intermediate code
used within the LLVM compiler infrastructure. It has gained popularity due to
its ability to support powerful optimization and transformation, and because
converting code to LLVM IR enables the use of LLVM’s comprehensive analysis and
transformation framework.
Among various binary lifters, the SoK study evaluated four representative
tools—McSema, RetDec, McToll, and BinRec—all of which translate binaries into
LLVM IR. In the study, each binary was lifted by all four lifters and evaluated
in terms of analysis suitability and functional correctness. For analysis
suitability, the study measured how well the lifted IR supports downstream
analyses. For functional correctness, the IR was recompiled, and its execution
results were compared with those of the original binary.
The results varied significantly depending on the IR style. In discriminability
analysis, high-level IR (HIR), as generated by RetDec and McToll, achieved
scores exceeding 75\%, whereas emulation-style IR (EIR), as generated by BinRec
and McSema, scored below 12\%. Conversely, in functionality tests, EIR exceeded
90\% correctness, while HIR scored below 25\%. As discussed in Section 1, EIR
tends to emulate machine code more closely, whereas HIR resembles
compiler-generated code. A key difference is that EIR employs a virtual stack,
which complicates analysis. Because the virtual stack mimics the program’s
behaviour, EIR preserves functional correctness effectively; however, the
absence of explicit variables and other high-level information reduces analysis
suitability.
In this work, we focus on transforming EIR into HIR. EIR preserves program
semantics well due to its emulated stack, but this same structure obscures the
recovery of high-level information required for analysis. Our objective is to
eliminate the virtual stack in EIR and transform the code into HIR while
preserving the semantics. This transformation aims to combine the functional
correctness of EIR with the analysis suitability of HIR.
For this study, we selected BinRec, a dynamic lifter, to generate EIR for
transformation. To achieve our goal, we identify two primary challenges:\\
1. Variable recovery \\
2. Type recovery \\

